\documentclass[a4paper,12pt,oneside]{jsarticle}

% \usepackage{document}
\usepackage{caption}
\usepackage[dvipdfmx]{graphicx}
% \setcounter{secnumdepth}{3}



\begin{document}
\begin{center}
  {\Large
    2025年 部内大会\\
    \vspace*{10truept}
		{\fontsize{15pt} {0cm}\selectfont マルチコプター部門  ユニークミッション}\\ % タイトル
		\vspace{30truept}
  }

  \vspace{15truept}	
  {\fontsize{12pt} {0cm}\selectfont 東京農工大学\hspace{5mm}航空研究会}\\
  \vspace{5truept}
  {\fontsize{12pt} {0cm}\selectfont 17代\hspace{5mm}大屋 悟士}\\ % 著者
  \vspace{20truept}
\end{center}
\hrulefill
\tableofcontents

\newpage
\section{反復横跳び}
\paragraph{成功条件}
\begin{itemize}
  \item 10秒以内に左右それぞれ2mの幅を計8回反復横跳びをすること
\end{itemize}

\paragraph{終了条件}
\begin{itemize}
  \item 次のミッションがコールされること
  \item 成功条件を満たすこと
  \item「反復横跳び」のコールから10秒以内にカウントが8回未満のとき
\end{itemize}

\paragraph{ミッションプロセス}
\begin{enumerate}
  \item 「反復横跳び」とコールする
  \item 2m間隔で設定された3本の線を超えるたびに1回カウントする
\end{enumerate}

\paragraph{付記}
\begin{itemize}
  \item 線を超える順は「真ん中,左,真ん中,右,真ん中,左, ...」というようにすること
  \item 最初に越える線は真ん中に限らない
  \item 「真ん中,左,真ん中,左」のように同じ方向の線を連続で超えてもカウントしない
\end{itemize}

\paragraph{点数}
\begin{itemize}
  \item 成功条件を満たしたときユニークミッション点として300点を与える
\end{itemize}

\newpage
\section{高度ビッタチャレンジ}
\paragraph{成功条件}
\begin{itemize}
  \item 指定された高度の±30cm以内に機体を10秒間ホバリングさせること
\end{itemize}
\paragraph{終了条件}
\begin{itemize}
  \item 次のミッションがコールされること
  \item 成功条件を満たすこと
\end{itemize}
\paragraph{点数}
\begin{itemize}
  \item 成功条件を満たしたときユニークミッション点として800点を与える
\end{itemize}
\paragraph{付記}
\begin{itemize}
  \item ホバリングすべき高度は主審により1から5mの範囲の整数値で指定する
  \item 高度維持を主目的としたソフトウェアによる制御は認めない
  \item 競技者は競技終了後速やかにFCのログを用いて指定範囲の高度でホバリングできたことを示すこと
  \item ログ上でホバリングの開始が明確にわかるように何らかの工夫をすること
\end{itemize}

\newpage
\section{耐全発故障制御}
\paragraph{成功条件}
\begin{itemize}
  \item 動力喪失後に不時着した機体が再離陸後ミッションエリア上空に移動すること
\end{itemize}

\paragraph{終了条件}
\begin{itemize}
  \item 次のミッションがコールされること
  \item チームメンバーが機体に触れること
\end{itemize}

\paragraph{ミッションプロセス}
\begin{enumerate}
  \item ミッションエリア内の高度2m以上を飛行中に「パワーオフ」のコールとともにすべてのプロペラを動力駆動しない状態にする
  \item 離着陸エリア内の任意の場所に接地し静止する
  \item 動力用モーターを再始動させ,離陸後ミッションエリア上空に移動する
\end{enumerate}

\paragraph{得点}
\begin{itemize}
  \item 成功条件を満たしたときユニークミッション点として700点を与える
\end{itemize}

\newpage
\section{ナイフエッジ}
\paragraph{成功条件}
\begin{itemize}
  \item 機体の上面が地面に対して垂直になっている状態で5秒以上(端数切り捨て)の飛行を行うこと
\end{itemize}

\paragraph{終了条件}
\begin{itemize}
  \item 次のミッションがコールされること
  \item 成功条件を満たすこと
\end{itemize}

\paragraph{ミッションプロセス}
\begin{enumerate}
  \item 機体の上面が地面に対して垂直な状態で「ナイフエッジ」のコールがされる
  \item 機体の上面が地面に対して垂直な状態を維持して5秒以上の飛行を行う
\end{enumerate}

\paragraph{機体上面の定義}
\begin{itemize}
  \item 機体審査用紙に添付された三面図に存在している面のうち、競技開始時バーティポートに駐機している状態で最も上に来る面
  \item 上記の機体上面の定義を適用した際に上面が一意に定まらない場合は大会当日に審判が決定する
\end{itemize}
\paragraph{得点}
\begin{itemize}
  \item 成功条件を満たしたときユニークミッション点として500点を与える
\end{itemize}
\end{document}
